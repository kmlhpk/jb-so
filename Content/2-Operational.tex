\hiddenchapter{\large{Operational Standing Orders}}
\ChapterTitle[\thechapter]{Operational Standing Orders}
\renewcommand{\thesection}{\textbf{O\arabic{section}}}
%%% See Preamble for an explanation of both of these %%%
\hiddensection{\large{Roles and Responsibilities of JCR Executive Committee Members}}
\SectionTitle{\thesection}{Roles and Responsibilities of JCR Executive Committee Members}
\begin{enumerate}
    \item Policy of the JCR
    \begin{enumerate}
        \item A new major policy of the JCR shall always be agreed at a JCR Meeting unless the policy
        decision must be made before the next scheduled Meeting.
        \item A JCR Emergency Meeting can be called, at the discretion of the JCR Chair, to discuss any
        issue that either cannot be delayed until the next JCR General Meeting and / or is deemed
        to be of sufficient importance to deserve immediate attention.
        \item A minimum of 24 hours’ notice must be given before an emergency meeting,
        and a quoracy count must be taken before the meeting begins.
        \item The quorum for JCR General and Emergency Meetings shall be 10\% of voting members,
        but will not be accounted for unless requested by any member present at the meeting.
        \item At least 20 members must be present at any JCR meeting. The Chair must count
        20 members before the beginning of any meeting.
    \end{enumerate}
    \item Membership Dues
    \label{it:Membership Dues}
    \begin{enumerate}
        \item All JCR members shall pay an agreed membership fee. The JCR Membership Levy shall be agreed by the JCR in the penultimate JCR meeting of Term 3 every academic year, chargeable to 1st year Undergraduates. This levy confers membership of the JCR for the first three years of an undergraduate degree. In particular circumstances, the cost of membership from the second year on-wards for late joiners shall be a proportion of the JCR Membership Levy and confer a relevant number of years of membership, as agreed by the JCR Finance Committee.
        \item Membership fees for Mature Undergraduates, 4th Year Undergraduates, Postgraduates, part
        time students and students on the Erasmus scheme shall be agreed jointly by the JCR and MCR
        and are set out in the Memorandum of Understanding between the two common rooms, which
        shall be available on the JCR website.
        \item In the event that a JCR member ceases to be a student at Josephine Butler College before their
        term of paid JCR membership expires, a member shall be entitled to a refund from the JCR
        dependant on the number of complete years remaining of their membership term. This refund
        shall be at a rate of one third of the JCR Membership Levy as was in effect at the time of the JCR
        member’s matriculation to the university per complete unused year of membership. Refunds shall
        only be payable where a request for a refund is made in writing by a JCR member to the current
        JCR President, and such requests will only be accepted if made within 2 calendar months of the
        JCR member ceasing to be a student at Josephine Butler College.
        \item JCR members who restart one or multiple years of study, either by retaking a number of years
        within their degree or by changing to a different degree programme, shall be charged an
        additional membership fee. This fee shall be equal to one third of the current JCR Membership
        Levy per restarted year, and confers the relevant number of additional years of JCR membership.
    \end{enumerate}
    \item Modification of Membership Fees
    \label{it: Modification of Membership Fees}
    \begin{enumerate}
        \item Membership Fees can only be modified by an agreed resolution at straight majority at a quorate JCR Meeting following the presentation of a motion by the finance committee.
        \item Any changes made to the Membership Fee cannot affect any membership dues already paid; it
        can only affect individuals who will in the future apply for membership of the JCR.
    \end{enumerate}
    \item JCR Awards
    \begin{enumerate}
        \item -	The JCR President will use their discretion when making the awards, although it is expected that they shall be conferred as follows:
        \begin{enumerate}
            \item \textbf{Vote of Thanks}: to JCR members who have demonstrated a significant contribution
            to the JCR over the course of a year.
            \item \textbf{Recognition of Outstanding Contribution} to the JCR: to JCR members who have
            demonstrated a significant contribution to the JCR over a number of years or have
            demonstrated a highly impressive contribution in the course of a single year.
            \item \textbf{Honorary Life Membership} shall normally only be conferred to individuals who have
            contributed exceptional commitment, diligence and excellence in services to the JCR
            throughout their college life.
            \item \textbf{The Community Award} may be granted to a person, persons or organisation external
            to the JCR in recognition of contributing outstanding services towards the JCR in
            meeting their objective
        \end{enumerate}
       \item Nominations shall be considered by outgoing executive committee. The President will chair this committee and have a deciding vote should any tie occur between the voting members.
        \item Nominators must submit thorough reasons for their nomination and may nominate themselves or others. 
        \item Nominations may be made anonymously.
        \item Nominees should be recognised at the earliest opportunity once the requisites of each award is met.
        \item If the nominee is on the panel, they shall be asked to leave the discussion and shall not be entitled to a vote.

    \end{enumerate}
\end{enumerate}

\hiddensection{\large{JCR Meetings}}
\SectionTitle{\thesection}{JCR Meetings}
\label{sc: JCR Meetings}
\begin{enumerate}
    \item Procedure for JCR Meetings
    \begin{enumerate}
        \item General JCR Meetings will adhere to the following format:
        \begin{enumerate}
            \item \textbf{Approval of Minutes} from the Previous Meeting, and any requests for changes.
            \item \textbf{Officer Reports}:
            \begin{enumerate}
                \item Each Exec officer will give a written report and short oral report on what
                they have been doing since the last meeting.
                \item JCR members will be given the opportunity to ask / submit in writing to the Chair, any question pertaining to an officer’s role but does not fulfil the criteria of a ‘Question of Officer’s Conduct.’
            \end{enumerate}
            \item \textbf{Motions}:
            \begin{enumerate}
                \item Motions will be made up of the following three components: Notes (current situation), Believes (Why it should be changed, and Resolves (How the situation should be changed.).
            \end{enumerate}
            \item Members may submit \textbf{Procedural Motions} which are as follows: Motion to
            \begin{enumerate}
                \item Move to a vote
                \item Move to an online referendum/election
                \item Move to a secret ballot vote
                \item Suspend the vote until the next meeting
                \item Amend the motion
                \item Call the end of the meeting
                \item Call a break
                \item Temporarily remove the Chair (to be replaced by a Vice-Chair)
                \item Object to any procedure
                \item In order to object to a procedural motion, a JCR member may call out “Objection.” It will then be taken to a vote by show of hands.
                \item Point of Information – factual information that would aid discussion.
                \item Point of Order – statement that something is out of order or unconstitutional.
                \item Point of Personal Privilege – environmental factors affecting persons (e.g. difficulty hearing)
            \end{enumerate}
            \item \textbf{Items of Discussion}, which includes any and all matters that could not be classified in any of the other categories above but deserve their own slot on the agenda and can result in a vote on the options available.
            \item \textbf{Elections}
            \begin{enumerate}
                \item Any outstanding non-executive positions will be up for election (see O-6.2).
                \item Each candidate must abide by the rules for the method of the role for which they are applying (see \ref{Election Methods}.
            \end{enumerate}
            \item \textbf{Any Other Business}, which includes Any last-minute items which could not be included in the agenda or any further
            matters arising from the Floor will fall into this category.
            \item Any voting which will take place online will be in the Single Transferable Voting (STV) format.
            \item All JCR meetings will be chaired by the JCR Chair. Where the Chair is unable to fulfil their duties, see Section \ref{sc:Chair} of Committees and Job Descriptions.
        \end{enumerate}
    \end{enumerate}
    \item Ruling Motions in Order
    \begin{enumerate}
        \item A motion will be deemed out of order if it violates any of the below criteria. A motion must:
        \begin{enumerate}
            \item Be factually accurate and unambiguous.
            \item Be in accordance with the Standing Orders.
            \item Be legal.
            \item Have followed the proper channels in financial matters before being submitted.
        \end{enumerate}
        \item If a motion is deemed to be out of order, Democracy Committee must discuss with the Proponent
        the rephrasing of the motion so that it could be deemed to be in order.
        \item If the proponent is unhappy with the proposed revisions and Democracy Committee cannot accept the motion without the revisions, the motion will not appear as a motion on the agenda. The proponent may commence the Formal Appeals Process (as outlined in \ref{it:Appeal}) if they so wish.
    \end{enumerate}
    \item Items for Discussion
    \begin{enumerate}
        \item An Item for Discussion may be submitted when a JCR member feels a situation needs to be changed but there are a number of solutions available.
        \item An Item for Discussion must fairly and objectively outline the current situation, problem, and the options available.
        \item The options will then be voted upon using either;
        \begin{enumerate}
            \item A hand vote at a JCR meeting using the ‘First Past the Post’ voting system. 
            \item A secret ballot at a JCR meeting using the STV voting format.
            \item An online referendum using the STV voting format.
        \end{enumerate}
    \end{enumerate}
    
\end{enumerate}

\hiddensection{\large{JCR Executive Committee}}
\SectionTitle{\thesection}{JCR Executive Committee}
\begin{enumerate}
    \item Meetings of the JCR Executive Committee
    \begin{enumerate}
        \item The JCR Chair shall chair Executive Committee meetings.
        \item A Vice-chair shall take minutes of Executive Committee meetings and be a non- voting member.
        \item The quorum for Exec meetings shall be 7 voting members.
        \item There will be only one vote per exec position in situations where the position is being held jointly by two people.
        \item The Executive Committee-elect shall be invited to attend the last weekly meetings of the Epiphany term as non-voting observers.
        \item An Emergency Meeting of the Executive Committee may be convened and a temporary JCR Chair and Vice-chair appointed from Democracy Committee as necessary, as long as the quorum requirement for the Executive Committee meetings is satisfied and every effort is made to notify all members of the Executive Committee and accommodate as many as possible.
        \item Ordinary members of the JCR can request to attend JCR Executive Committee meetings by emailing the Chair and be invited at his/her discretion.
        \item The terms of office for Executive Committee members shall normally run from the first day of the Easter holidays following their election until the last day of Epiphany Term in the forthcoming year, save for sabbatical officers of the JCR. 
    \end{enumerate}
    \item Jointly held Exec Positions
    \begin{enumerate}
        \item If after being elected jointly, one person resigns or becomes incapacitated and is unable to continue to fulfil the requirements of their role, and the other wishes to continue, an Item for Discussion will be presented to the JCR at the next General Meeting to determine whether the JCR will support the remaining person on his or her own or to hold a new election.
    \end{enumerate}
    \item Minute Taking Procedure
    \begin{enumerate}
        \item All minutes should be signed (digitally if appropriate) by all members of Democracy Committee present at the meeting and approved at the next meeting.
        \item Where members of Democracy Committee are required to take minutes of meetings as part of their role, the following guidelines should be used:
        \begin{enumerate}
            \item Due to their importance the minutes should include the date, time and venue of the meeting, the name of the committee sitting, the names and capacities of all those attending and, if appropriate, the point at which anyone attending left the meeting; apologies for absence; the name of the Chair; and the date, time and venue of the next meeting (if known).
            \item The body of the minutes should include;
            \begin{enumerate}
                \item  The precise wording of any motion, together with the name of the proposer
                and (optionally) the seconder of the motion;
                \item Information upon which the decision was based;
                \item Details of the decision, i.e. who voted and how and, in the event of an equality of votes, if the Chair used a casting vote;
                \item The action required as a result of the motion;
                \item The names (or a shorthand for the names) of the people who are responsible
                for implementing the decision;
                \item A summary of the discussion on each item of business.
            \end{enumerate}
            \item Redaction may be used sparingly where an individual’s private and personal life is in need of protection; or disclosure of information would seriously compromise the negotiating position of the individual who disclosed it.
            \item Democracy Committee holds the decision in regard of what redactions to make but may consult the JCR Welfare Officers for advice.
            \item In the event of any redactions, Democracy Committee should hold a full, un- redacted copy. \item Un-redacted copies of minutes will be made available to members of the JCR present during the
            redacted discussion, the current JCR Chair and the Vice-chairs at the discretion of the Chair.
            \item Un-redacted copies of any discussion will be treated as confidential, and must be destroyed after use, with the exception of the copy held by Democracy Committee.
            \item At the discretion of the JCR Chair, un-redacted copies of the minutes may be made available to relevant parties.
            \item A register will be kept by the JCR Chair of each individual who has accessed un- redacted copies of any minutes. The JCR Chair shall be excluded from this register.
            \item If anyone who was in attendance is unable to agree that the draft minutes are an accurate record of the meeting, they should draw the matter to the attention of the chair before they are approved at the next meeting. If after discussion the attendee is still unable to agree then their dissension should be formally noted and recorded as a postscript to the minutes before they are approved.
        \end{enumerate}
    \end{enumerate}
\end{enumerate}

\hiddensection{\large{JCR Committees}}
\SectionTitle{\thesection}{JCR Committees}
\begin{enumerate}
    \item Roles and Organisation
    \begin{enumerate}
        \item A committee of the JCR differs from a society in that it acts to benefit the JCR as a whole, and not only those who are members of it.
        \item Committees, alongside their specific roles within the JCR community, should also strive to continually encourage increased involvement and membership.
        \item Any committee may be asked to submit a brief written report about their activities for the next General JCR meeting by any member of the JCR. The request must be made at least 10 days before the next meeting to give the committee sufficient time to prepare the report. This report should be sent to the JCR Chair to be included in the agenda 7 days before a general JCR Meeting, and the committee chair may also be called upon to answer any relevant questions asked of them at a JCR meeting by a JCR member. Failure to write a report without suitable reason when requested to do so may invoke a ‘Question of Officers Conduct’ under \ref{sc:Complaints}.
    \end{enumerate}
    \item Resignation From Committees
    \begin{enumerate}
        \item For all JCR Committees (excluding members of exec where mandated to attend), if a person holding an elected position fails to attend three consecutive meetings of the Committee without making any apologies, or four consecutive meetings with apologies, they will be considered to have stepped down from the position, which will then go up for election at the next General/Emergency Meeting of the JCR, or for internal roles at the next General/Emergency Committee Meeting.
        \item The JCR Chair and Vice-chairs must be notified of any resignations as soon as possible.
    \end{enumerate}
\end{enumerate}

\hiddensection{\large{JCR Societies}}
\SectionTitle{\thesection}{JCR Societies}
\begin{enumerate}
    \item Roles and Organisation
    \begin{enumerate}
        \item A JCR Society is a recreational organisation within the JCR which benefits its members but not necessarily the whole JCR or college.
        \item A society is not required to specifically produce anything for the college but if appropriate should assist with JCR events and campaign weeks.
        \item Society Presidents are elected at compulsory AGMs, which should be recorded and sent to the JCR Societies Officer. They must then report to the JCR Societies Officer each term about the progress of the society and a report on the progress of the budget when appropriate, as well as any changes to the structure of the society. It is compulsory that they attend the freshers’ fair or send a member of the society as a delegate.
        \item Societies should constantly strive to increase and improve membership throughout the year and should attempt to have at least three meetings in a term, unless a valid reason is given.
        \item If a society is not run at all in a term, without a valid reason, it will be considered without a president and then an EGM must be held, and minutes must be taken.
    \end{enumerate}
    \item Ratification
    \begin{enumerate}
        \item A recreational group may be ratified by the JCR as a society if the JCR believes that it currently has (and will continue to have) a sustainable membership and will continue to benefit its members. The JCR Societies Officer must approve the ratification before it goes up in the meeting.
        \item Once a society is ratified it may apply for JCR funding or charge a membership fee. Their budget must then be maintained throughout the year by the society president or a member of the society’s exec.
        \item A ratified society may have official association with the Josephine Butler College JCR, and thus may use Josephine Butler within their name.
        \item If the JCR finds that a society has been inactive for at least a year then the society will be de-ratified and will lose the benefits above.
        \item The assets of a de-ratified society must be transferred to another JCR group or back to the JCR as a whole.
        \item There must be a list of ratified societies which is kept and maintained by the Societies Officer.
    \end{enumerate}
\end{enumerate}

\hiddensection{\large{JCR Projects}}
\SectionTitle{\thesection}{JCR Projects}
\begin{enumerate}
    \item Roles and Organisation
    \begin{enumerate}
        \item Projects are student groups which have a specific focus on fulfilling our JCR Values within and external to our membership. 
        \item Each project shall have a leader elected by the membership to ensure continuity with our external relationships. 
        \item Projects should strive to continually encourage increased involvement and membership.  
        \item Projects shall be supported by the Vice-President who will offer oversight and support. 
    \end{enumerate}
    \item Ratification
    \begin{enumerate}
        \item A project may be ratified if the JCR believes that it currently has (and will continue to have) a sustainable membership and will continue to promote the JCR’s values. The Vice-President must approve the ratification before it goes up in the meeting.  
        \item Once a project is ratified it may apply for JCR funding or charge a membership fee. Funding shall be proportionate to the number of JCR members who are part of the project. Their budget must then be maintained throughout the year by the Project Leader or by a delegate from the Project. 
        \item A ratified Project may have official association with the Josephine Butler College JCR, and thus may use Josephine Butler within their name.  
        \item If the JCR finds that a project has been inactive for at least a year then the society will be de-ratified and will lose the benefits above.  
        \item The assets of a de-ratified project will be transferred back to the JCR. 
        \item There must be a list of ratified projects which is kept and maintained by the Vice-President.
    \end{enumerate}
\end{enumerate}

\hiddensection{\large{Election Methods}}
\SectionTitle{\thesection}{Election Methods}
\label{Election Methods}
\begin{enumerate}
    \item Method 1: 
    \begin{enumerate}
        \item This method shall be used for members of the executive committee or other positions where a higher level of accountability is required than for other roles. 
        \item There shall be a period of campaigning (see campaign rules in \ref{Method 1 Campaign Regulations}).
    \end{enumerate}
    \item Method 1a: 
    \begin{enumerate}
        \item This method shall be used for the election of the sabbatical officers.
        \item Additional to method 1, candidates shall complete a challenge as set by the outgoing officer.
    \end{enumerate}
    \item Method 2:
    \begin{enumerate}
        \item This method shall be used for most JCR positions.
        \item There shall be no period of campaigning.
        \item Candidates must nominate themselves for election by either emailing the Chair ahead of the meeting or by putting their name down on a sheet to be made available by Democracy Committee during the meeting.
        \item When called to hust, husts for non-executive election may last no longer than 3 minutes except following intervention from the Democracy Committee.
        \item In a contested election, whilst one candidate husts, the other candidates will be required to leave the room. At the end of the hustings, all candidates will be invited back into the room for questions.
        \item If the candidate cannot make it to the meeting, they must email their name, the position they wish to stand for along with a hust of no longer than three minutes to the JCR Chair. This can then be read out by a representative.
        \item All questions must be addressed to all candidates and may be submitted in writing to the JCR Chair.
        \item For all contested elections not carried out online, candidates must obtain a simple majority of votes cast in a quorate JCR Meeting. For all uncontested elections, candidates require a super majority of 2/3rds of votes cast.
    \end{enumerate}
        \item Method 2a:
    \begin{enumerate}
        \item This method shall be used for representative roles and expands on the rules for method 2.
        \item Members should only vote if they self-identify as a member of the community represented by the role. 
        \item All elections should take place online.
    \end{enumerate}
    \item Method 3:
    \begin{enumerate}
        \item This method shall be used for officers assistants.
        \item There shall be no period of campaigning.
        \item The election of assistant officers shall be done via an application and selection process which is coordinated by the officer to be assisted.
        \item The process shall be overseen by the President and FACSO.
        \item The election of these roles will be announced in the JCR meeting as outlined in the Schedule of Elections \ref{Election Schedule}.
    \end{enumerate}
\end{enumerate}


\hiddensection{\large{Method 1 Campaign Regulations}}
\SectionTitle{\thesection}{Method 1 Campaign Regulations}
\label{Method 1 Campaign Regulations}
\begin{enumerate}
    \item Nominations
    \begin{enumerate}
        \item Any current full member of the JCR has the right to stand for a method 1 election so long as they can fulfil their term of office and meet the criteria for their position as outlined in the Standing Orders.
        \item Individuals should nominate themselves by emailing the JCR Chair stating their name and the position they are running for.
        \item The candidate should speak to the incumbent officer about the roles and responsibilities they are volunteering to undertake prior to nominating themselves.
        \item As soon as a person has nominated themselves, the Chair will send them a copy of the campaign and election rules and regulations as laid out in the Standing Orders. Candidates are expected to attend a meeting with Democracy Committee the night nominations close. Where a poster is to be used, this must be brought to the meeting for approval. No posters produced or submitted after this time will be acceptable campaign material.
        \item The nomination period ends 2 days prior to the campaign period starting, to allow Democracy Committee time to approve any campaign materials
        \item Individuals must meet with the JCR Sabbatical Officers, and their predecessor, before nominations close. Should the predecessor not be available, the Sabbatical Officers can unilaterally grant permission for a candidate to run. Individuals may also need to meet with others for a specific role, as stated in the role's description.
        
    \end{enumerate}
    \item Campaign Rules and Regulations
    \label{it:rules}
    \begin{enumerate}
        \item Election campaigning may take place during the week before voting is scheduled to open.
        \item Candidates are allowed to produce any of the following campaign materials:
        \begin{enumerate}
            \item A Poster
            \item A 400-Word Written Manifesto
            \item One Promotional Image
            \item One Two-Minute Video 
        \end{enumerate}
        \item All written manifestos must contain a statement saying "Approved by Democracy Committee" at the bottom, and all materials must contain a link to the JCR website where all materials will be shared
        \item Individuals are not allowed to make or share any campaign materials beyond those approved by Democracy Committee 
        \item Individuals must not make any defamatory remarks about other candidates, or discourage voting for a specific candidate; candidates’ election campaigns should focus on their individual merits.
        campaigns should focus on their individual merits.
        \item Democracy Committee actions:
        \begin{enumerate}
            \item Place an A4 copy of each candidate’s poster on the Democracy Committee notice board in the Bar.
            \item Upload a PDF version of each poster onto the JCR website.
            \item Email all campaign posters in one go so as to ensure that Livers’ Out are reached in the campaign process.
            \item Maintain, update, and keep current the elections section of the JBJCR website 
        \end{enumerate}
        \item Executive Committee actions:
        \begin{enumerate}
            \item Place a poster (preferably laminated if possible, for maximum longevity) with a link and QR code to the aforementioned Elections section on the notice board in each livers-in kitchen, as well as any existing and future Democracy Committee notice boards.
        \end{enumerate}
        \item Candidates’ campaign materials must not include other individuals or candidates, and candidates for separate positions (ie. different elections, or two individuals not running as a pair) may not campaign together
        \item Candidates are forbidden from spending their own money on their campaigns.
        \item Candidates for JCR President are expected to hust for a maximum of 5 minutes. Before the hust, the current president will give a short outline of everything the position entails.
        \item Candidates for other executive positions are expected to hust for a maximum of 4 minutes. Before the hust, the incumbent officer will give a short outline of everything their position entails.
        \item Candidates may not under any circumstance or pretence interfere in the voting procedure.
        \item Candidates may not under any circumstance mention their relationship to a named person and their role during hustings otherwise they will be given a verbal warning. If this persists further, then vote deductions will take place by the discretion of the chair.
        \item Candidates may not coerce, pressure, bribe, or blackmail anyone into voting for them 
        \item Any breach of the Campaign Rules and Regulations listed in \ref{it:rules} by the candidate will be immediately investigated by the JCR Chair and may result in appropriate disciplinary action, which includes the possibility of exclusion of that candidate from the election. Ignorance of the rules and regulations will not be accepted as an excuse.
    \end{enumerate}
    \item Election Rules and Regulations
    \begin{enumerate}
        \item Each full JCR member is entitled to a vote with the exception of members of the Democracy Committee.
        \item A tie will result in a revote.
        \item The voting format for all JCR Exec positions will be Single Transferable Voting (STV). 
        \item Candidates may call for a recount.
        \item Candidates may nominate a witness on their behalf when counting the ballots.
        \item Ballot papers or other records of the election must be kept for the rest of the academic year.
        \item Quoracy for elections is 10\% of the JCR.
        \item Voting for Exec positions will be open online for 72 hours starting as soon as possible on the night of the hustings (or as soon as possible after).
        \item If any position remains vacant nominations will be reopened in time for the next JCR meeting
        \item If an acting officer chooses to step down, nominations will be reopened at the nearest opportunity but with no less than 3 days for nominations and 7 days for canvassing before husting and voting opens.
        \label{it:Reopen Election}
        \item Election results and figures for all Exec positions will be announced in the Bar after the polls have closed. Following the announcement, results will be emailed to all JCR members within 24 hours. Candidates have the option to request to be informed of the results prior to them becoming public knowledge.
        \item In the event on an uncontested election, the running candidate requires 2/3 of the vote, otherwise nominations will be reopened for the next JCR meeting.
    \end{enumerate}
\end{enumerate}

\hiddensection{\large{Schedule Of Elections}}
\SectionTitle{\thesection}{Schedule of Elections}
\label{Election Schedule}
\begin{enumerate}
    \item The election method (\ref{Election Methods}) is identified using the notation M.X where X is the method number.
    \item Executive Elections (M.1)
    \begin{enumerate}
        \item First JCR Meeting of Epiphany Term
        \begin{enumerate}
            \item Chair
            \item Social Chair 
            \item Treasurer 
            \item JCR JBs Officer
        \end{enumerate}
        \item Second JCR Meeting of Epiphany Term
        \begin{enumerate}
            \item President
            \item Sports Officer 
            \item Societies Officer 
            \item Welfare Officers
        \end{enumerate}
        \item Third JCR Meeting of Epiphany Term
        \begin{enumerate}
            \item Vice President 
            \item Publicity Officer 
            \item International Officer 
            \item Technical Directors
            \item FACSO
        \end{enumerate}
    \end{enumerate}
    \item JCR Officer Elections
    \begin{enumerate}
        \item 1st Meeting of Michaelmas
        \begin{enumerate}
            \item DSU Rep (Individual) (M.2)
            \item Year Out Rep(s) (Individual or Pair) (M.2)
            \item Tour Reps (2 Individuals) (M.2)
            \item Quizmaster (Individual) (M.2)
            \item Mole Master (Individual) (M.2)
            \item Photography Officer (Individual) (M.2)
            \item Assistant Photography Officer(s) (M.3)
            \item Assistant Mound Editor(s) (M.3)
            \item Livers-In Rep (Individual) (M.3)
        \end{enumerate}
        \item 2nd Meeting of Michaelmas
        \begin{enumerate}
            \item Butler Gala Chair(s) (Individual or Pair) (M.2)
            \item Assistant Quizmaster (Individual) (M.2)
            \item Butler Talks Chair(s) (Individual or Pair) (M.2)
            \item Wardrobe Committee Chair(s) (Individual or Pair) (M.2)
            \item Ethnic Minorities Rep(s) (Individual or Pair) (M.2a)
            \item Students of Faith Rep (Individual) (M.2a)
            \item Women's Rep (Individual) (M.2a)
            \item Assistant Charity Committee Chair(s) (M.3)
            \item Charity Committee Treasurer (M.3)
            \item Assistant DSU Rep (Individual) (M.3)
        \end{enumerate}
        \item 3rd Meeting of Michaelmas
        \begin{enumerate}
            \item Head Open Day Rep(s) (Individual or Pair) (M.2)
            \item Summer Ball Chairs (Pair) (M.2)
            \item Butler Day Chairs (Pair) (M.2)
            \item Hedgehog Protectors (2 Individuals or Pair) (M.2)
            \item Working Class Students Rep (Individual) (M.2a)
            \item Assistant Green Committee Chair(s) (M.3)
            \item Assistant Open Day Rep(s) (M.3)
        \end{enumerate}
        \item 1st Meeting of Epiphany
        \begin{enumerate}
            \item Chair (Individual) (M.1)
            \item Treasurer (Individual) (M.1)
            \item Social Chair (Individual) (M.1)
            \item JBs Officer (Individual) (M.1)
            \item Food Services Officer (Individual) (M.1)
            \item Yearbook Editor (Individual) (M.2)
            \item Policy Officer (Individual) (M.1)
            \item Assistant JBs Officer(s) (x3) (M.3)
        \end{enumerate}
        \item 2nd Meeting of Epiphany
        \begin{enumerate}
            \item JCR President (Individual) (M.1a)
            \item Welfare Officers (2 Individuals)  (M.1)
            \item Societies Officer (Individual) (M.1)
            \item Sports Officer (Individual) (M.1)
            \item Vice-Chair (3 Individuals) (M.2)
            \item Assistant Social Chair(s) (M.3)
            \item Vice-Treasurer (Individual) (M.2)
        \end{enumerate}
        \item 3rd Meeting of Epiphany
        \begin{enumerate}
            \item JCR Vice-President (Individual) (M.1)
            \item International Officer (Individual) (M.1)
            \item Publicity Officer (Individual) (M.1)
            \item FACSO (Individual) (M.1a)
            \item Head Livers-Out Rep (Individual) (M.2)
            \item Students with Disabilities Rep (Individual) (M.2a)
            \item Students with Disabilities Neurodiversity Rep  (Individual) (M.2a)
            \item Assistant Societies Officer(s) (M.3)
            \item Assistant Publicity Officer (M.3)
            \item Assistant Sports Officer (M.3)
            \item Livers-Out Rep (Individual) (M.3)
            \item Assistant Welfare Officer (1 Male and 1 Female) (M.3)
        \end{enumerate}
        \item 1st Meeting of Easter
        \begin{enumerate}
            \item Head Freps (Pair) (M.2)
            \item Music Committee Chairs(s) (Individual or Pair) (M.2)
            \item Stash Coordinator (Individual) (M.2)
            \item Mound Editor(s) (Individual or Pair) (M.2)
            \item Volunteering Coordinator (Individual) (M.2)
            \item LGBT+ Gender Identity Rep (Individual) (M.2a)
            \item LGBT+ Orientation Rep (Individual) (M.2a)
            \item Assistant International Officer (2 Individuals) (M.3)
            \item Assistant Welfare Officer (3 Individuals) (M.3)
        \end{enumerate}
        \item 2nd Meeting of Easter
        \begin{enumerate}
            \item Winter Ball Chairs (Pair) (M.2)
            \item Green Committee Chair (Individual or Pair) (M.2)
            \item Charity Committee Chair(s) (Individual or Pair) (M.2)
            \item Arts Director (Individual) (M.2)
            \item Assistant Arts Director(s) (M.3)
            \item Gym Rep (Individual) (M.2)
            \item Careers and Alumni Rep (Individual) (M.2)
            \item Webmaster (2 individuals or Pair) (M.2)
            \item Finance Trustees (3 Individuals) (M.2)
            \item Intergenerational Project Leader (M.3)
            \item RT Projects Leader (M.3)
            \item Freps (M.3)
            \item Group Leader Freps (M.3)
            \item Take-a-break Freps (M.3)
        \end{enumerate}
    \end{enumerate}
    \item JCR Committee Elections
    \begin{enumerate}
        \item For the Social, Arts, Ball, Butler Day, Charities, Fashion Show, Green, Mound, Publicity, Projects, Ball and Technical Comm. JCR members simply must attend meetings to be considered members.
    \end{enumerate}
    \item Head Frep(s) and Head Open Day Rep(s) Elections
    \begin{enumerate}
        \item Candidates for these roles may produce one A4 poster, to be publicised only by Democracy Committee, to aid their campaign.
    \end{enumerate}
\end{enumerate}

\hiddensection{\large{Conduct of JCR Office Holders}}
\SectionTitle{\thesection}{Conduct of JCR Office Holders}
\begin{enumerate}
    \item Conduct of Officers
    \begin{enumerate}
        \item Executive, Non-Executive, and Committee Members of the JCR shall act in the best interests of the JCR when performing their duties.
        \item All Officers and Committee members of the JCR shall act in accordance with their requirements in the JCR Standing Orders and shall not abuse their position.
        \item Any full member of the JCR can call a motion of no confidence in a member of the JCR Executive Committee at any JCR General or Emergency meeting.
        \item If an executive officer of the JCR is unable to fulfil his or her duties for a period of a month or more during term time, for any given reason, the officer will be asked to resign and the position will be reopened for election in accordance with time-frame outlined in section \ref{it:Reopen Election}.
    \end{enumerate}
\end{enumerate}

\hiddensection{\large{Complaints and Discipline}}
\SectionTitle{\thesection}{Complaints and Discipline}
\label{sc:Complaints}
\begin{enumerate}
    \item Question of Officer's Conduct
    \begin{enumerate}
        \item A Question of Officer’s conduct may be submitted if a JCR member believes an officer is not fulfilling their duties to the best of their ability or not fully performing all aspects of their role as laid out in the Standing Orders.
        \item A Question of Officer applies to any JCR member holding an elected position in any JCR club, society, or committee, and must be submitted at least 7 days in advance of a JCR meeting. Emergency questions or queries can be submitted and acted upon at the Chair’s discretion.
        \item The Chair must inform the office holder of the submission as soon as possible. The proponent’s identity will be known only by the JCR Chair and will be kept completely confidential.
        \item The Question will be presented as a separate agenda item at the next JCR meeting and read out by the Chair to maintain anonymity. The Officer will then answer the question to the best of their ability within the meeting.
    \end{enumerate}
    \item Question of Officer's Conduct and Vote of Censure
    \begin{enumerate}
        \item A Vote of Censure may be submitted in conjunction with a Question of Officer, but will appear as a separate agenda item after the Question. The criteria and procedure in \ref{sc:Complaints} must be followed.
        \item A straight majority at a JCR meeting is required to uphold a vote of censure which will serve as an official warning to the officer. The officer can appeal the decision within 7 days by following the Formal Appeals Procedure as outlined in \ref{it:Appeal} of the Operational Standing Orders.
        \item If an officer receives a second vote of censure within the same term of office, then they shall be subject to an immediate Vote of No Confidence. If they subsequently receive a further Vote of Censure, then they will be subject to a further Vote of No Confidence.
        \item If members of the JCR are satisfied with the answer to the Question of Officer’s Conduct they can call for the Vote of Censure to be withdrawn through a procedural motion.
    \end{enumerate}
    \item Question of Officer's Conduct and Vote of No Confidence
    \label{it: QOO VONC}
    \begin{enumerate}
        \item A Vote of No Confidence may be submitted in conjunction with a Question of Officer but not on its own. The criteria and procedure must be followed and the Vote of No Confidence will appear as another separate agenda item after the Question of Officer.
        \item A straight majority at a quorate JCR meeting will remove this officer from the position in question. The officer may appeal the decision within 7 days by following the Formal Appeals Procedure as outlined in Section \ref{it:Appeal} the Standing Orders.
        \item If members of the JCR are satisfied with the answer to the Question of Officer’s Conduct they can call for the Vote of No Confidence to be withdrawn through a procedural motion.
    \end{enumerate}
    \item Grounds for Suspension or Exclusion from the JCR
    \label{it: Grounds for Suspension or Exclusion}
    \begin{enumerate}
        \item Grounds for suspension or exclusion from the JCR shall include but not be limited to:
        \begin{enumerate}
            \item Abuse or harassment of any kind directed at any member or group of members of the JCR.
            \item Intentional misappropriation of JCR funds.
            \item Bringing the JCR into disrepute.
            \item Displays of prejudice.
            \item Repeated and / or malicious breaches of the Standing Orders.
            \item Wilful damage and/or conduct likely to bring the JCR into disrepute, at any event organised or supported by the JCR.
            \item Intentional damage to one or more items of property of the JCR.
            \item Any other behaviour which unreasonably compromises the ability of the JCR to deliver its
            aims and objections as set out in \ref{sec: Purpose} of the Permanent Standing Orders.
        \end{enumerate}
    \end{enumerate}
    \item Procedure for Suspension and Exclusion
    \begin{enumerate}
        \item Any member of the JCR may bring a request for a suspension or exclusion of another JCR member from the JCR. The proponent must inform the JCR Chair who will either take it to the next Exec meeting or will call an emergency Exec meeting.
        \item The proponent will then be asked to attend the meeting and present their case. The JCR member in question will then be asked to come and present their case.
        \item Any decision of temporary suspension or permanent exclusion of any individual shall be made by decision of the JCR Executive Committee by virtue of a straight majority vote at a quorate JCR Executive Committee Meeting once the situation has been brought forward to a member of the Executive Committee.
        \item Both the accuser and the defender will be given the chance to argue their case.
        \item The JCR Chair shall inform the suspended or excluded individual of their suspension or exclusion in
        writing or via email within twenty-four hours of the decision being made.
        \item The suspended or excluded individual may appeal against the decision once only, unless subsequent investigation reveals new and compelling evidence.
        \item Details of suspensions or exclusions shall never be made publicly available by the JCR save for in the event of an Appeal in which case details of the case may be used to serve to justify the decision, at the discretion of the JCR Chair.
    \end{enumerate}
    \item Procedure for Phase One of a Formal Appeal
    \label{it:Appeal}
    \begin{enumerate}
        \item In the event of a formal appeal, the JCR Chair shall call an Emergency JCR Executive Committee Meeting where the JCR Chair shall outline the details of the appeal, which must include both the original decision against which the appeal in made and the details of the opposition as made by the appealing member.
        \item The JCR Chair shall then facilitate free and fair discussion of this matter by the Executive Officers
        \item During this discussion, the JCR Chair will invite the proponent and the defendant of the appeal to attend a period of the discussion in order to put forward their argument and answer questions from the Executive Officers.
        \item When the general consensus of the group is clear, or if the discussion has run its course and it is apparent that there is no group consensus, the JCR Chair shall bring the discussion to a close via holding a formal straight majority vote, the decision of which will be final, unless Phase Two of a Formal Appeal is instigated, according to the procedure outlined in \ref{it:Formal Appeal Phase 2}.
        \item The JCR Executive Committee may also decide to vote to uphold or reject the appeal or instigate Phase 2.
        \item In the event of a tie, the JCR Chair shall have the casting vote.
        \item The quorum figure for a Phase One Formal Appeal meeting shall be 8 voting members.
        \item If, upon the conclusion of Phase One of a Formal Appeal, any member of the JCR is not satisfied with the outcome of the appeal, they may request the instigation of Phase Two.
        \item In this event, the appealing member of the JCR must inform the JCR Chair in writing or by email of their request for the instigation of Phase Two within seven days of the appeal’s outcome.
        \item If no member of the JCR registers with the JCR Chair their request for the instigation of Phase Two within seven days, the outcome of the Formal Appeal, as laid out in the Standing Orders, will be final.
        \item If the JCR Chair has a personal interest in the appeal, then the JCR Executive Committee will appoint a temporary Chair to take responsibility for the Formal Appeals Process.
    \end{enumerate}
    \item Procedure for Phase Two of a Formal Appeal
    \label{it:Formal Appeal Phase 2}
    \begin{enumerate}
        \item In the event that a formal appeal is forwarded to Phase Two, the JCR Chair shall add this to the agenda for a JCR General Meeting or where necessary, call an Emergency General Meeting.
        \item The JCR Chair shall prepare a motion in which the background information of the matter in question is presented and the resolution of which is only to revoke the original decision made against which the Phase Two Formal Appeal is made. During the meeting, the JCR Chair shall present in a fair and balanced fashion the aforementioned background information of the matter in question.
        \item The JCR Chair shall invite one person to make a proposition speech (i.e. pro revocation) and one person to make an opposition speech (i.e. against revocation), where possible identified by the JCR Chair before the meeting. The JCR Chair shall invite the Proposition and Opposition speakers to address the arguments, following which an open discussion shall be had, and a vote shall be taken.
        \item The motion shall pass (i.e. revocation) by virtue of a simple majority vote in its favour.
        \item In the event of a tie, the JCR Chair shall have the casting vote.
        \item The quorum figure for a Phase Two Formal Appeal motion shall be 10\% of voting members. 
        \item The outcome of this meeting shall be final.
    \end{enumerate}
    \item Complaints Procedure
    \begin{enumerate}
        \item If a JCR member is dissatisfied in their dealings with the JCR, one of its members or officers, they may use the JCR Complaints Procedure.
        \item Firstly they should informally approach the relevant JCR officer/member who shall attempt to deal with the complaint to the satisfaction of the complainant.
        \item Where the complainant remains dissatisfied, or chooses not to approach the relevant JCR Officer/member, they should approach the JCR President who shall attempt to deal with the complaint, and bring the matter to the attention of the Executive where appropriate.
        \item Where the complainant remains dissatisfied, or feels unable to approach the JCR President, or where the complaint concerns the JCR President, they should approach the JCR Chair who shall attempt to resolve the matter to the satisfaction of the complainant.
        \item Where the complainant remains dissatisfied they can follow one of the following courses of action:
        \begin{enumerate}
            \item For a complaint about a JCR Officer they may submit a 'Question of Officers Conduct' and
            if it is grave enough a 'Motion of No Confidence' (see O-9.3).
            \item For a complaint about the behaviour of a JCR member they can approach a member of the Executive Committee about the 'Suspension or Expulsion from the JCR' (see \ref{it: Grounds for Suspension or Exclusion}).
            \item For a complaint about a decision that has been made they can lodge a formal appeal (see sections \ref{it:Appeal} and \ref{it:Formal Appeal Phase 2}).
        \end{enumerate}
    \end{enumerate}
    \item JCR Disciplinary Panel
    \begin{enumerate}
        \item The JCR Disciplinary Panel shall convene when any member of the JCR has been alleged to have brought the JCR into disrepute. They will meet to deal with certain offences as an alternative to a college disciplinary panel, but serious offences shall be referred straight to college at the panels’ discretion.
        \item The disciplinary panel shall comprise the JCR President and JCR Vice President. It shall be chaired by the JCR Chair and minuted by a Vice-chair nominated by the Chair. In the case of a deadlock, the JCR Chair will be called on to make a casting vote.
        \item The minutes of disciplinary panels shall be treated as redacted minutes.
        \item Minutes of the panel will be made available to the current JCR President, and to members of any future disciplinary panels concerning the same individual.
        \item If one of the panel is unavailable or has a conflict of interest, a replacement shall be appointed from the executive committee at the discretion of the JCR Chair
        \item Procedure of the JCR Disciplinary Panel:
        \begin{enumerate}
            \item The accused shall be offered the right to a college disciplinary hearing instead.
            \item The JCR Chair will outline the reason for the disciplinary hearing and will invite the accused to present their case.
            \item The remainder of the panel will be invited to ask questions to establish the facts of the case.
            \item When the panel are satisfied that they have enough information to reach an informed decision, they will agree on a suitable course of action; this can include dismissing the case, community service, letters of apology, further investigation or recommending permanent or temporary exclusion from the JCR to the exec committee.
            \item The accused shall be informed of their right to appeal the decision to a college disciplinary hearing as the normal JCR appeals process does not apply in this instance.
        \end{enumerate}
    \end{enumerate}
    \item Suspension or Exclusion from the JCR
    \label{it: Suspension or Exclusion}
    \begin{enumerate}
        \item Members of the JCR may be suspended or excluded from the JCR if their conduct has been deemed to have been grossly inappropriate. Section \ref{it: Grounds for Suspension or Exclusion} offers suitable ground without limitations.
        \item Members of the JCR may be suspended temporarily for a specified period or excluded permanently from membership of the JCR which includes involvement in any events and/ or making use of any services of the JCR (save for JCR Welfare Services, of which they may make use at the discretion of the Welfare Officers and/ or Assistant Welfare Officer).
        \item Suspension from the JCR can be made by a straight majority vote at a JCR Executive Committee Meeting, save for in the event of an emergency, when the President may exercise the right to impose temporary suspension, which will normally be for no more than the length of time till the next quorate JCR Executive Committee Meeting.
    \end{enumerate}
    \item Registering a Formal Appeal
    \begin{enumerate}
        \item A formal appeal may be registered in writing or by email to the JCR Chair:
        \begin{enumerate}
            \item By any member of the JCR against the legitimacy of any decision, interpretation or
            election made by the JCR.
            \item By a member of the JCR against their own or that of another member of the JCR’s suspension or exclusion under \ref{it: Suspension or Exclusion}.
            \item By any member of the JCR against any other incident or activity so judged sufficient by the individual to warrant an appeal.
        \end{enumerate}
        \item All formal appeals against JCR decisions must be explained fully and clearly to the JCR Chair, and the lodger of the appeal must make specific reference to the decision against which they are appealing.
        \item Appeals against a decision may be made only once unless additional evidence comes to light.
    \end{enumerate}
    \item Timescale and Notification
    \begin{enumerate}
        \item An appeal shall be registered with the JCR Chair within seven days of the occurrence of the event (or in the circumstance of a series of events, the final event in that specific series) about which the appeal is made unless in exceptional circumstances.
        \item Upon receiving a registration for a formal appeal, the JCR Chair shall inform the member of the JCR who has registered an appeal of safe receipt of this registration as soon as is practical.
        \item The JCR Chair shall keep the appealing member of the JCR informed and aware of the status of their appeal on a regular basis; and shall update this individual no less than once per week of the progression of their appeal.
        \item When a decision has been made during a formal appeal, the JCR Chair must inform the appellant as quickly as is practical.
    \end{enumerate}
\end{enumerate}

\hiddensection{\large{Availability of JCR Documentation}}
\SectionTitle{\thesection}{Availability of JCR Documentation}
\begin{enumerate}
    \item Standing Orders
    \begin{enumerate}
        \item It is the responsibility of the JCR Chair to maintain the definitive copy of the Standing Orders, and to make this document available during JCR Meetings, JCR Formal Appeals and at all other appropriate times.
    \end{enumerate}
    \item Other Documentation
    \begin{enumerate}
        \item The release of any other documents of the JCR may be made at the discretion of the JCR President in discussion with the relevant JCR Exec Officer(s), save for any membership details (please note that the Data Protection Act may overrule this stipulation).
    \end{enumerate}
    \item Membership Details
    \begin{enumerate}
        \item Only general membership figures may be released by the JCR Chair to anybody other than a College Official. Membership details must remain strictly confidential to the JCR Chair and JCR President, save for a request by a College Official.
    \end{enumerate}
    \item Adgendas and Minutes
    \begin{enumerate}
        \item Agendas and Minutes from all JCR General Meetings and Executive and Finance Committee Meetings, as well as those from important committee meetings (i.e. those where potentially contentious decisions are made) shall be published online on the JCR Website (with the exception of minutes from a closed session).
        \item In the case of Agendas, these shall be produced and published at least 72 hours before a JCR General Meeting (by the JCR Chair). Agendas for JCR Committee meetings should ideally be produced 24 hours before the meeting is to take place.
        \item In the case of Minutes, these shall be produced and published at most 72 hours following a JCR General Meeting. Minutes for JCR Committee meetings should ideally be produced 72 hours following the meetings
    \end{enumerate}
    \item Website Access
    \begin{enumerate}
        \item The JCR is required to maintain a Documentation section on the JCR Website, where all relevant JCR Documentation (e.g. the Standing Orders, Agendas and Minutes) shall be made available to all JCR members.
    \end{enumerate}
\end{enumerate}


