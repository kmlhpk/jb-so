\clearpairofpagestyles
\renewcommand{\pagemark}{{\usekomafont{pagenumber}{\thepage\ / \pageref{LastPage}}}}
\ohead*{\vspace{-1cm}\hfill} %For if we want a thing to appear in the top-left corner of every page
\ofoot*{\pagemark}
\pagenumbering{arabic}
\setcounter{page}{1} %%% This is only to be included in the first content file to restart the page numbers using a different format %%%

\hiddenchapter{\large{Permanent Standing Orders}}
\ChapterTitle[\thechapter]{Permanent Standing Orders}
\renewcommand{\thesection}{\textbf{P\arabic{section}}}
%%% See Preamble for an explanation of both of these %%%

\hiddensection{\large{Preface}}
\SectionTitle{\thesection}{Preface}
\begin{enumerate}
    \item In these Standing Orders, the following terms shall be understood with the following meanings:
    \begin{enumerate}
        \item \textbf{JCR} refers to the Junior Common Room of Josephine Butler College
        \item \textbf{College} refers to Josephine Butler College
        \item \textbf{DSU} refers to Durham Students’ Union 
        \item \textbf{Standing Orders} shall refer collectively to these Permanent Standing Orders and the Operational Standing Orders of the JCR
        \item \textbf{DSO Framework} shall refer to the Durham Student Organisations Framework which was most recently approved by the University Council
    \end{enumerate}
    \item The JCR is the official student representative organisation of Josephine Butler College which operates within the Durham Student Organisation (DSO) Framework.
    \item In accordance with the Framework, the `parent body' of the JCR is the College.
\end{enumerate}

\hiddensection{\large{Status of the Standing Orders}}
\SectionTitle{\thesection}{Status of the Standing Orders}
\begin{enumerate}
    \item The JCR shall be operated in accordance with the DSO Framework and its Standing Orders. Should there be a conflict between the Framework and the Standing Orders the requirements of the Framework shall take precedence.
    \item The Standing Orders of the JCR exist to regulate the democratic structures and operational principles of the JCR and shall include these Permanent Standing Orders and the Operational Standing Orders of the JCR.
    \item Where there is a conflict between the Permanent Standing Orders and the Operational Standing Orders, the Permanent Standing Orders shall take precedence. In the event the JCR Chair judges there to be a conflict between two articles within the Permanent or Operational Standing Orders, then the lower numbered of the conflicting articles will take priority; any parts of the higher numbered articles which do not conflict will remain in force.
    \item The DSO Framework and Standing Orders shall be consulted and complied with in all JCR policies, procedures, rules, regulations, decisions and other such matters.
    \item  These Permanent Standing Orders may not be suspended. Any part of the Operational Standing Orders of the JCR may be suspended by a vote at a quorate JCR meeting. Any alternative arrangements established must comply with the DSO Framework and these Permanent Standing Orders.
    \item These Standing Orders remain the intellectual property of the JCR.
    \item The JCR shall establish a committee, chaired by the JCR President, to review the Standing Orders no less frequently than once every five years.
\end{enumerate}

\hiddensection{\large{Amending the Standing Orders}}
\SectionTitle{\thesection}{Amending the Standing Orders}
\begin{enumerate}
    \item The Standing Orders may only be modified by a straight majority vote on an agreed resolution at a quorate JCR Meeting and following consultation with the DSO framework.
    \item Changes to Permanent Standing Orders may only be made by approval through a JCR referendum where quoracy is 10\%.
    \item Changes to Operational Standing orders may only be made by approval through JCR motion in accordance with the procedure set out in \ref{sc: JCR Meetings}
    \item It is assumed that changes will take effect at the end of the meeting, unless otherwise stipulated in the motion of change.
    \item It is the responsibility of the JCR Chair to effect any modifications to these Standing Orders within three full days of the JCR Meeting at which the resolution was satisfactorily agreed. % TODO lol this time limit is stupid
\end{enumerate}

\hiddensection{\large{Purpose of the JCR}}
\SectionTitle{\thesection}{Purpose of the JCR}
\label{sec: Purpose}
\begin{enumerate}
    \item The purpose of the JCR shall be:
    \begin{enumerate}
        \item To represent the views of the JCR membership to the College Council, College Officers and the wider College community; the academic departments and wider University services; appropriate local organisations and other relevant organisations.
        \item To ensure effective communication between JCR members and other bodies of the University.
        \item To provide social and cultural activities within the College, and to endeavour to cater for members’ social and cultural interests.
        \item To promote the maintenance of good health and welfare within the College and to endeavour to cater for the member’s health and welfare needs.
        \item To appropriately administer the finances of the JCR and to ensure JCR expenditure is made in the pursuit of the realisation of one or more of the JCR’s aims and objectives, and is procured at a competitive market rate.
    \end{enumerate}
    \item No member of the JCR shall support any activities which could bring the JCR into disrepute. % TODO I don't like that this point is in this section - it causes you to have unnecessary indentation by requiring that top P-4.1 point. Also I think we could do with re-wording it to add "... or go against its stated purposes" to the end, or something similar.  
\end{enumerate}

\hiddensection{\large{Membership of the JCR}}
\SectionTitle{\thesection}{Membership of the JCR}
\begin{enumerate}
    \item All student members of the College who have paid the appropriate membership fees are considered members of the JCR for the appropriate length of time as outlined in Section \ref{it:Membership Dues} of the Operations Standing Orders, except unless in the case of one or more of the following:
    \begin{enumerate}
        \item Their membership is formally suspended or annulled (consistent with the procedures of Section \ref{sc:Complaints} of the Operational Standing Orders);
        \item The student decides to opt out of their membership by letter to the JCR President. Full membership dues will be returned if this occurs with the first 30 days of Michaelmas term, if not Section \ref{it: Modification of Membership Fees} of the Operational Standing Orders shall apply.
    \end{enumerate}
    \item The identity of non-members shall not be made public knowledge and shall only be communicated to College Officers for appropriate purposes.
    \item  The administration and management of the system of membership is the responsibility of the JCR Chair and President.
    \item The JCR Chair will maintain a record of the members and non-members of the JCR in an electronic database.
    \item Non-members of the JCR may apply to join the JCR at any time by notifying the JCR Chair; and will be accepted if they meet the membership requirements (consistent with this Section) and have paid the appropriate JCR membership fee, as outlined in Section \ref{it: Modification of Membership Fees} of the Operational Standing Orders.
    \item All members of the JCR shall be required to pay a levy as outlined in the \ref{it: Modification of Membership Fees} of the Operational Standing Orders. 
    \item Only JCR members shall be entitled to vote on decisions of or to be an officer of the JCR. Non-members may at the discretion of the JCR be invited to attend events or use services provided by the JCR, but will have to pay a premium for services or an additional £5, £10 or £20 pounds depending on the size of the event.
    \item A person may request the JCR Chair to investigate the membership status of an individual, for example if it is suspected they are using JCR facilities without having paid membership dues, though this information will not be communicated to the person making the request.
    \begin{enumerate}
        \item Any action deemed necessary as a result of such a request should be decided by the JCR President and JCR Chair, but shall normally involve charging an appropriate proportion of membership levy up to and including 100\%.
    \end{enumerate}
\end{enumerate}

\hiddensection{\large{Business of the JCR}}
\SectionTitle{\thesection}{Business of the JCR}
\begin{enumerate}
    \item JCR members shall collectively be responsible for the operation and management of the JCR and any facilities, events and services it provides. All major decisions of the JCR shall normally be made at a JCR Meeting at which all members of the JCR are entitled to attend.
    \item All members of the JCR shall have an equal right to vote and to speak on any matter at any JCR Meeting. The Operational Standing Orders shall set out the procedures by which business is considered at JCR Meetings.
    \item The JCR shall have an Executive Committee which shall:
        \begin{enumerate}
            \item Act in the best intentions of the JCR at all times.
            \item Represent the JCR to the College and wider University community;
            \item Make decisions on behalf of the JCR and be responsible and accountable for them both individually and collectively.
            \item Follow the instructions of JCR Meetings, including, but not limited to being accountable and giving an Officer Report and answering questions from the floor.
            \item Be responsible for the day-to-day administration of the JCR, contributing to the decision- making process in both JCR and Executive meetings.
            \item Be responsible for making decisions on behalf of the JCR in the event of an emergency or where it is not possible to convene a full JCR Meeting to discuss the matter.
            \item Meet weekly during term-time, the minutes of these Meetings being available for inspection by any full JCR member following ratification (with the exception of redacted minutes).
            \item Attend all JCR Meetings unless there are extenuating circumstances, in which case apologies and a written report submitted to the JCR Chair before the meeting.
            \item Report to JCR Meetings on issues that have arisen since the last meeting, as well as accounting for any decisions made and any future plans.
            \item Maintain relevant sections of the JCR website,
            \item Make themselves available for Exec week, Frep week and Freshers’ week i.e. the 3 weeks prior to the start of Michaelmas term.
            \item Undergo a sensitivity training program conducted by the Equality and Diversity Committee (see C8.8) 
before beginning their roles.

        \end{enumerate}
        \item The Executive Committee shall be chaired by the JCR Chair and its membership will include the following:
        \begin{enumerate}
            \item President
            \item Finance and Community Services Officer
            \item Vice-President
            \item Treasurer
            \item Welfare Officer
            \item Social Chair
            \item Publicity Officer
            \item Societies Officer
            \item Sports Officer
            \item Services Officer
            \item Bar Steward
        \end{enumerate}
        \item The Chair shall be a non-voting member of the Executive Committee
        \item The JCR may delegate responsibilities to committees and officers as set out in the Operational
Standing Orders.
        \item Matters which are deemed by a JCR meeting or the Executive Committee to be of significance shall be referred to a JCR meeting or referendum of all members. Referenda shall be considered quorate if 10\% of the members vote and a motion shall be passed on the basis of a simple majority.
        \item Changes to these Permanent Standing Orders shall always be subject to a referendum.
\end{enumerate}

\hiddensection{\large{Appointment and removal of JCR Officers}}
\SectionTitle{\thesection}{Appointment and removal of JCR Officers}
\begin{enumerate}
    \item All JCR Officers shall be elected by the JCR. In all elections Re-open Nominations (RON) shall be included as a candidate.
    \item  Executive Officers (i.e. those officers of the JCR who are members of the Executive Committee) shall be appointed using Single Transferable Voting in accordance with the rules set out by the Electoral Reform Society.
    An election shall be considered quorate if 10\% of members have voted.
    \item  Non-executive Officers (i.e. those officers of the JCR who are not members of the Executive Committee) shall normally be appointed at a JCR meeting. Should the JCR or the Executive Committee agree, a Non-Executive Officer may be appointed using the mechanism established for Executive Officers
    \item  Any non-sabbatical officer of the JCR may stand down from their role by writing to the JCR Chair, who shall report this to the next JCR meeting.
    \item  Should an officer of the JCR be considered to have fallen short of fulfilment of the duties assigned to them, a quorate JCR meeting may agree a motion of censure against that officer.
    \item  Should a JCR officer continue to fall short of fulfilment of the duties assigned to them following a motion of censure, or in the event that the actions of an officer are considered to be serious misconduct, a quorate JCR meeting may agree on the basis of a secret ballot a motion of no- confidence in that officer, unless a motion for another type is raised. In the event a motion of no- confidence is agreed, the officer shall be required to stand down.
    \item  Any vacancy which arises shall be filled at the earliest opportunity by an election conducted in the manner normal for that post. The vacancy shall be filled for the remainder of the original term of office. Any officer appointed in this manner shall be eligible to stand for re-election for a full term of office at the normal time should they continue to meet any conditions for election to that role.
    \item  The following special conditions apply to the appointment of sabbatical officers of the JCR:
    \begin{enumerate}
        \item Following election their appointment shall only be valid following the signing of a
contract of employment with the University.
        \item Sabbatical officers may resign from their post by giving written notice of one term to the Executive Committee and the University Director of HR.
        \item If a sabbatical officer is accused of committing a serious offence that falls within the University’s definition of gross misconduct (as set out in the University’s Disciplinary Regulation) the JCR Executive must immediately consider a motion of no-confidence, and if agreed, bring this forward to the next JCR Meeting or convene an Emergency JCR Meeting as appropriate
        \item In the event of a motion of no-confidence in a sabbatical officer, a panel will be convened to consider the case for dismissal from office. The membership of the panel shall be agreed by the members of the JCR Executive Committee in consultation with the Director of HR (or their deputy).
        \item Should a sabbatical officer be dismissed from Office they have the right of appeal. The appeal process will be determined by the JCR Executive in consultation with the HR Director or his/her Deputy.
        \item Where there is a vacancy for a sabbatical officer, it shall be filled for the remainder of the term of office by a non-sabbatical officer elected in the normal manner. An individual appointed in these circumstances shall be eligible to stand for re-election for a full term of office.
    \end{enumerate}
\end{enumerate}

\hiddensection{\large{JCR Finances}}
\SectionTitle{\thesection}{JCR Finances}
\label{sc: JCR Finances}
\begin{enumerate}
    \item In accordance with the DSO Framework, the JCR President is ultimately accountable for the JCR’s finances and for ensuring that the JCR remains in a sound financial position.
    \item  The JCR Treasurer shall be responsible to the JCR President for all financial transactions of the JCR, and shall advise the JCR President, Executive Committee, Finance Committee and JCR as a whole on financial matters. The Treasurer shall be responsible for the preparation of the JCR budget and accounts, for liaising with the Colleges Accounts Team and the Head of College, and reporting financial information to the JCR, College and University as appropriate
    \item  The JCR Treasurer shall by default be the Accountable Officer for all cash handling operations as defined in the University’s Cash Control Principles. For certain operations, the President may jointly be/replace the Treasurer as the Accountable Officer. A log of all cash handling operations performed by the JCR and the appropriate Accountable Officer for each shall be kept the JCR Treasurer, and the Cashier’s Office within the University’s Finance Department notified whenever changes are made to it.
    \item Other officers may be delegated specific financial responsibilities when deemed appropriate by the JCR Treasurer or President. Their work will be overseen by the Treasurer or President accordingly.
    \item  Further financial responsibilities of the JCR President and Treasurer shall be set out in the Committees and Officer Role Descriptions of the Standing Orders within their respective descriptions.
    \item Indemnities:
    \begin{enumerate}
        \item Every executive officer, non-executive officer, committee member and member of staff of the JCR shall be entitled to be indemnified out of the assets of the JCR against all losses or liability which they may incur in or about the execution of his or her office or otherwise in relation thereto.
        \item  No officer, committee member or member of staff shall be liable for any loss, damage or misfortune which may happen to or be incurred by the JCR in the execution of the duties of their office duties;
        \item  Nothing in this clause should affect their liability for the consequences of any negligent action on their part.
    \end{enumerate}
\end{enumerate}

\hiddensection{\large{Complaints and Appeals}}
\SectionTitle{\thesection}{Complaints and Appeals}
\begin{enumerate}
    \item Members of the JCR are expected in normal circumstances and in the first instance to discuss informally and as soon as possible any matters of dissatisfaction with the individual(s) concerned. Where this is not possible, members should discuss the matter with the JCR President or the JCR Chair. Where the JCR President and the JCR Chair could both be considered to have a conflict of interest, initial discussions should be held with the Vice-President, Treasurer, and Welfare Officer first. If that is not possible the issue shall be referred to a member of the University Staff nominated for the purpose by the Head of College.
    \item If, following such discussions, a member of the JCR wishes to appeal against a decision of a committee or officer of the JCR:
    \begin{enumerate}
        \item They may appeal to a JCR meeting by writing to the JCR Chair. If the member or the JCR Chair consider that it would be inappropriate for the matter to be considered collectively by the JCR the matter shall be referred in writing to the Exec. If they consider that action inappropriate they may refer to the Head of College who shall endeavour to resolve the dispute.
        \item Following consideration of an appeal at a JCR meeting, if either the complainant or the JCR remain dissatisfied the matter may be raised in writing with the Head of the College who shall refer the matter to the College Council for resolution.
    \end{enumerate}
    \item Further detail on the exact procedures for Complaints and Appeals shall be located within Section \ref{sc:Complaints} of the Operational Standing Orders.
\end{enumerate}